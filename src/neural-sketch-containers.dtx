% ----------------------------------------------------------------
%  neural-sketch-containers.dtx
% ----------------------------------------------------------------
% \iffalse
%<*driver>
\documentclass{ltxdoc}
\EnableCrossrefs
\CodelineIndex
\RecordChanges

\begin{document}
\DocInput{\jobname.dtx}
\end{document}
%</driver>
% \fi

\ProvidesFile{neural-sketch-containers.dtx}%
[2025/01/22 v0.1 neural-sketch-containers]

% \title{neural-sketch-containers.sty\\[1ex]
%        Container logic for neural-sketch}
% \author{Vincenzo Buono}
% \date{\today}
% \maketitle
%
% \StopEventually{}

% implementation ~~~~~~~~~~~~~~~~~~~~~~~~~~~~~~~~~~~~~~~~~~~~~~~~~~~~~~~~~~~ <<<
%<*package>
%    \begin{macrocode}
\ExplSyntaxOn


% ~~~~~~~~~~~~~~~~~~~~~~~~~~~~~~~~~~~~~~~~~~~~~~~~~~~~~~~~~~~~~~~~~~~~~~~~~~~~~~
% Custom pgf layers
% ~~~~~~~~~~~~~~~~~~~~~~~~~~~~~~~~~~~~~~~~~~~~~~~~~~~~~~~~~~~~~~~~~~~~~~~~~~ <<<
\pgfdeclarelayer{background}
\pgfdeclarelayer{interim}
\pgfdeclarelayer{foreground}
% layer registration ~~~~~~~~~~~~~~~~~~~~~~~~~~~~~~~~~~~~~~~~~~~~~~~~~~~~~~~ <<<
\pgfsetlayers{background,interim,main,foreground}


% ~~~~~~~~~~~~~~~~~~~~~~~~~~~~~~~~~~~~~~~~~~~~~~~~~~~~~~~~~~~~~~~~~~~~~~~~~~~~~~
% Pass-through
% store unkown keys to pass-through
% ~~~~~~~~~~~~~~~~~~~~~~~~~~~~~~~~~~~~~~~~~~~~~~~~~~~~~~~~~~~~~~~~~~~~~~~~~~ <<<
\prop_new:N \l_nsk_container_unknown_prop

% ~~~~~~~~~~~~~~~~~~~~~~~~~~~~~~~~~~~~~~~~~~~~~~~~~~~~~~~~~~~~~~~~~~~~~~~~~~~~~~
% Position Parser
% ~~~~~~~~~~~~~~~~~~~~~~~~~~~~~~~~~~~~~~~~~~~~~~~~~~~~~~~~~~~~~~~~~~~~~~~~~~ <<<
\cs_new_protected_nopar:Npn \nsk__container_parse_pos_value:nn #1#2
{
% #1 = the user’s "right=1cm of X" or "above=2mm of Y" ...
% #2 = the container’s padding dimension (e.g. 3mm)

% Save the raw pos string to \l_nsk_container_pos_tl
\tl_set:Nx \l_nsk_container_pos_tl {#1}

% Split at '=' (only if the user wrote e.g. "right=1cm of X").
\seq_set_split:Nnn \l_tmpa_seq {=}{#1}

% If we have exactly 2 items, item 1 is "right" and item 2 is "1cm of X".
\int_compare:nNnT { \seq_count:N \l_tmpa_seq } = {2}
{
% The direction (left/right/above/below)
\tl_set:Nx \l_tmpa_tl { \seq_item:Nn \l_tmpa_seq {1} }
\tl_trim_spaces:N \l_tmpa_tl

% The distance+node portion
\tl_set:Nx \l_tmpb_tl { \seq_item:Nn \l_tmpa_seq {2} }
\tl_trim_spaces:N \l_tmpb_tl

\tl_show:N \l_tmpa_tl
\tl_show:N \l_tmpb_tl

% split at " of "
\seq_set_split:Nnx \l_tmpb_seq {of}{\l_tmpb_tl}


\int_compare:nNnT { \seq_count:N \l_tmpb_seq } = {2}
{
% item 1 = "1cm", item 2 = "X"
\tl_set:Nx \l_tmpc_tl { \seq_item:Nn \l_tmpb_seq {1} }
\tl_trim_spaces:N \l_tmpc_tl
\tl_set:Nx \l_tmpd_tl { \seq_item:Nn \l_tmpb_seq {2} }
\tl_trim_spaces:N \l_tmpd_tl
\tl_show:N \l_tmpc_tl
\tl_show:N \l_tmpd_tl


% Convert "1cm" to a dimension variable
\dim_new:N \l__nsk_tmp_dim
\dim_set:Nn \l__nsk_tmp_dim {\tl_use:N \l_tmpc_tl}
\dim_show:N \l__nsk_tmp_dim
% Now add the container's padding
\dim_add:Nn \l__nsk_tmp_dim  { #2 }
\dim_show:N \l__nsk_tmp_dim

% ~~~~~~~~~~~~~~~~~~~~~~~~~~~~~~~~~~~~~~~~~~~~~~~~
% String buld
% Rebuild "right=(dist+pad) of X"
% ~~~~~~~~~~~~~~~~~~~~~~~~~~~~~~~~~~~~~~~~~~~~ <<<
\tl_set:Nx \l_nsk_container_pos_tl
{
	\tl_use:N \l_tmpa_tl
	= \dim_use:N \l__nsk_tmp_dim
	\c_space_tl of \c_space_tl
	\tl_use:N \l_tmpd_tl
}
}
}
}


% ~~~~~~~~~~~~~~~~~~~~~~~~~~~~~~~~~~~~~~~~~~~~~~~~~~~~~~~~~~~~~~~~~~~~~~~~~~~~~~
% Main nsk / container Primitive
% ~~~~~~~~~~~~~~~~~~~~~~~~~~~~~~~~~~~~~~~~~~~~~~~~~~~~~~~~~~~~~~~~~~~~~~~~~~ <<<
\keys_define:nn {nsk / container}
{
	% ref-id ~~~~~~~~~~~~~~~~~~~~~~~~~~~~~~~~~~~~~~ <<<
	% todo: passed through
	id .tl_set:N = \l_nsk_block_id_tl,
	id .initial:n = { },

	% styles ~~~~~~~~~~~~~~~~~~~~~~~~~~~~~~~~~~~~~ <<<
	border-type .code:n =
		{
			\tl_if_eq:nnTF {#1}{none}
			{
				\tl_set:Nn \l_nsk_container_border_type_tl {draw=none}
			}
			{
				\tl_set:Nn \l_nsk_container_border_type_tl {#1}
			}

		},
	border-type .initial:n = {solid},
	border-type .default:n = {solid},

	border-color .tl_set:N = \l_nsk_container_border_color_tl,
	border-color .initial:V = \g__nsk_style_container_border_color_tl,
	border-color .default:V = \g__nsk_style_container_border_color_tl,

	fill .tl_set:N = \l_nsk_container_fill_tl,
	fill .initial:V = \g__nsk_style_container_fill_tl,
	fill .default:V = \g__nsk_style_container_fill_tl,

	padding .dim_set:N = \l_nsk_container_padding_tl,
	padding .initial:V = \g__nsk_style_container_padding_dim,
	padding .default:V = \g__nsk_style_container_padding_dim,

	shift-x .fp_set:N = \l_nsk_container_x_fp,
	shift-x .initial:V = \g__nsk_style_container_x_fp,
	shift-x .default:V = \g__nsk_style_container_x_fp,

	shift-y .fp_set:N = \l_nsk_container_y_fp,
	shift-y .initial:V = \g__nsk_style_container_y_fp,
	shift-y .default:V = \g__nsk_style_container_y_fp,

	% natural positioning using positioning library
	pos .code:n =
		{
			\nsk__container_parse_pos_value:nn {#1} {\l_nsk_container_padding_tl}
		},
	pos .initial:n = { },
	pos .default:n = { },

	rotate .fp_set:N = \l_nsk_container_rotate_fp,
	rotate .initial:V = \g__nsk_style_container_rotate_fp,
	rotate .default:V = \g__nsk_style_container_rotate_fp,

	scale .fp_set:N = \l_nsk_container_scale_fp,
	scale .initial:V = \g__nsk_style_container_scale_fp,
	scale .default:V = \g__nsk_style_container_scale_fp,


	% -- The special pass-through key for arbitrary options.
	% forward unknown props to nsblocks
	unknown .code:n =
		{
			% Store unknown key in the property list, using the content of \l_keys_key_tl as a string
			\prop_put:Nxx \l_nsk_container_unknown_prop
			{ \tl_use:N \l_keys_key_tl }
			{ #1 }
		},

}

% ~~~~~~~~~~~~~~~~~~~~~~~~~~~~~~~~~~~~~~~~~~~~~~~~~~~~~~~~~~~~~~~~~~~~~~~~~~~~~~
% Public Interface
% - (user-facing)
% ~~~~~~~~~~~~~~~~~~~~~~~~~~~~~~~~~~~~~~~~~~~~~~~~~~~~~~~~~~~~~~~~~~~~~~~~~~ <<<
\NewDocumentCommand \nskContainer { O{} +m }
{
	\group_begin:

	% (a) Clear the property list so it doesn't retain any old unknown keys
	% \prop_clear_new:N \l_nsk_container_unknown_prop

	% (b) Parse recognized container keys + store unknown keys
	\keys_set:nn {nsk / container}{#1}

	% (c) Now call \nskGroup, passing the same #1 so that shift-x, rotate, etc.
	%     get recognized by nsk / group (or remain unknown for that domain).
	\nskGroup[
		shift-x=\fp_use:N \l_nsk_container_x_fp,
		shift-y=\fp_use:N \l_nsk_container_y_fp,
		rotate=\fp_use:N \l_nsk_container_rotate_fp,
		scale=\fp_use:N \l_nsk_container_scale_fp,
		pos=\tl_use:N \l_nsk_container_pos_tl,
	]
	{
		% user content
		#2
	}

	% (d) Build an \nskBlock that fits the bounding box, plus any unknown keys
	%     stored in \l_nsk_container_unknown_prop. We'll do a small helper:
	\nsk__container_draw_block:

	\group_end:
}

% block expander ~~~~~~~~~~~~~~~~~~~~~~~~~~~~~~~~~~~~~~~~~~~~~~~~~~~~~~~~~~~ <<<
\cs_new_protected_nopar:Npn \nsk__expand_block_aux:n #1
{

	\begin{pgfonlayer}{background}
		\nskBlock [
			% this are computed by bounding box
			width=0pt, height=0pt,
			shadow=false,
			text-north=\nskBlockID,
			#1,
		]
	\end{pgfonlayer}
}

% ~~~~~~~~~~~~~~~~~~~~~~~~~~~~~~~~~~~~~~~~~~~~~~~~~~~~~~~~~~~~~~~~~~~~~~~~~~~~~~
% Variant Generation :V
% ~~~~~~~~~~~~~~~~~~~~~~~~~~~~~~~~~~~~~~~~~~~~~~~~~~~~~~~~~~~~~~~~~~~~~~~~~~ <<<
\cs_generate_variant:Nn \nsk__expand_block_aux:n { V }
\cs_generate_variant:Nn \nskBlockFromList:V { V }

% ~~~~~~~~~~~~~~~~~~~~~~~~~~~~~~~~~~~~~~~~~~~~~~~~~~~~~~~~~~~~~~~~~~~~~~~~~~~~~~
% Main draw
% ~~~~~~~~~~~~~~~~~~~~~~~~~~~~~~~~~~~~~~~~~~~~~~~~~~~~~~~~~~~~~~~~~~~~~~~~~~ <<<
\cs_new_protected_nopar:Npn \nsk__container_draw_block:
{
% We'll build a token list of block options
\tl_clear_new:N \l_tmpa_tl

% put recognized container keys:
\tl_put_right:No \l_tmpa_tl { type=rectangle, }
\tl_put_right:No \l_tmpa_tl { fill=\l_nsk_container_fill_tl, }
\tl_put_right:No \l_tmpa_tl { border-color=\l_nsk_container_border_color_tl, }
% use tikz-ops keys
\tl_put_right:No \l_tmpa_tl
{ tikz-opts={fit=(\l_nsk_group_bounds_name_tl), inner\space sep=\l_nsk_container_padding_tl,
			\l_nsk_container_border_type_tl}, }


% forward unknown keys as-is
\prop_map_inline:Nn \l_nsk_container_unknown_prop
{
	\tl_put_right:Nn \l_tmpa_tl
	{
		{##1} = {##2},
			}
	}

	\nsk__expand_block_aux:V \l_tmpa_tl
}

\ExplSyntaxOff
%    \end{macrocode}
%</package>
\endinput
