\documentclass{standalone}
\standaloneconfig{border=2mm}

\usepackage{../src/neural-sketch}
\usepackage{../src/neural-sketch-groups}
\usepackage{../src/neural-sketch-containers}
\usepackage{../src/neural-sketch-bridge}



\begin{document}

% \tikzset{
% % bridging path/.initial=arc,
% % bridging span/.initial=8pt,
% % bridging gap/.initial=4pt,
% bridge/.style 2 args={
% spath/split at intersections with={#1}{#2},
% spath/insert gaps after
% components={#1}{\pgfkeysvalueof{/tikz/bridging span}},
% spath/join components upright
% with={#1}{\pgfkeysvalueof{/tikz/bridging path}},
% spath/split at intersections with={#2}{#1},
% spath/insert gaps after
% components={#2}{\pgfkeysvalueof{/tikz/bridging gap}},
% }
% }
% \tikz[overlay] { \path[spath/save global=arc] (0,0) arc[radius=1cm,
% 			start angle=180, delta angle=-180]; }
%
\begin{tikzpicture}[]
	\nskBlock[type=rectangle,tikz-opts={dashed}]
	\nskBlock[id=ugly, type=rectangle,tikz-opts={dashed}, fill=purple, x=-1.5,
		% text-north={You are ugly},
		text-north-loc  = {above},
	]
	\draw[->] (ugly.south east) -- ++(0, -.5) --++(1, 0) -- (rectangle1.south);
	% \draw[thick, -latex] (ugly.north) to[bend left=90] (rectangle1.north);
	\draw[thick, -latex] (ugly.north) -- ++(0, 1) -| (rectangle1.north);
	\nskBlock[type=rectangle, border-type=solid, fill=blue, x=1.5]
	\nskBlock[
		type        = rectangle,
		fill        = red,
		y           = 1.5,
		tikz-opts       = {rounded corners=3pt},
		text-north      = {North Label},
		text-north-loc  = {above},
		text-north-style  = {color=red, font=\footnotesize},
		% text-south      = {South Label},
		text-south-loc  = {below left},
		text-south-style  = {color=blue, font=\footnotesize},
		text-east       = {East Side Note},
		text-east-loc   = {right},
		text-east-style  = {color=orange, font=\footnotesize},
		text-west       = {West Side Note},
		text-west-loc   = {above left},
		text-west-style  = {color=green, font=\footnotesize},
		text-center     = {Center},
		text-center-loc = {}, % no special placement (ust center)
		text-center-style = {font=\footnotesize}, % no special placement (ust center)
	]

	\nskGroup[
		shift-x=-1,
		shift-y=3.5,
		rotate=45,
		scale=1,
		group-style={draw=red, ultra thick},
	]{
		% child content: \nskBlock or raw TikZ
		\nskBlock[type=rectangle, border-color=red, fill=none, x=0]
		\nskBlock[type=rectangle, border-color=red, fill=none, x=1.5]
	}

	% Now let's enclose some content in a container
	\nskContainer[
		shift-x=8,
		rotate=30,
		fill=green!20,
		border-color=green,
		border-type=dashed,
		id=potato2,
		fill=red!50,
		padding=5pt,
	]{
		% child content
		\nskBlock[ x=0,  y=0, fill=orange ]
		\nskBlock[ x=1.5,y=0, fill=yellow!50 ]
		\nskBlock[ x=0, y=-1.5, fill=blue!50 ]
	}
	\draw[->] (potato2.south) -- ++(0, -2);

	\nskBlock[id=boxA, x=0, y=-3.5]
	\nskBlock[id=boxB, x=2, y=-3.5]

	% A normal arrow
	\nskConnect[
		from=boxA.east,
		to=boxB.west,
		% arrow-style={ultra thick, ->},
		arrow-style={thick, ->},
		color=blue,
		bridging=true,
		bridging-style=over,
	]

	\nskBlock[id=boxC, x=1, y=-5]
	\nskBlock[id=boxD, x=1, y=-2]

	\nskConnect[
		from=boxC.north,
		to=boxD.south,
		arrow-style={ultra thick, -latex},
		color=red,
		bridging=true,
		bridging-style=under,
	]

	\nskDoBridging

\end{tikzpicture}


hello41

\end{document}
